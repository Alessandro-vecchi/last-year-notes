\documentclass[openright, twoside, twocolumn]{report}
\usepackage{amsmath}
\usepackage[usenames,dvipsnames,table]{xcolor}
\usepackage{amssymb}
\usepackage{tikz}
\usepackage{listings}
\usepackage{quotchap}
\usepackage{epigraph}
\usepackage{etoolbox}
\usepackage{lipsum}
\usepackage[T1]{fontenc}
\usepackage{xparse}
\usepackage{hyperref}
\usepackage{amsmath}
\usepackage{amsfonts}
\usepackage{mathtools}
\usepackage{amsthm}
\usepackage{bookmark}
\usepackage{thmtools}
\usepackage[para]{footmisc}
\usepackage{booktabs}


%────────────────────────────────────────────────────────────────────────────────────────────────────────────────────────────────────────────────────
% 
% Theorems 
% 
%────────────────────────────────────────────────────────────────────────────────────────────────────────────────────────────────────────────────────

% Main theorem style
\declaretheorem[style=definition,numberwithin=section,thmbox=S]{theorem,definition}
\declaretheorem[thmbox=M,numberwithin=section]{lemma,corollary,proposition}

\declaretheorem[shaded={rulecolor=ProcessBlue,
rulewidth=2pt, bgcolor={rgb}{1,1,1}}]{remark,observation}

\declaretheorem[style=remark,shaded={rulecolor=Red,
rulewidth=2pt, bgcolor={rgb}{1,1,1}}]{example}

\declaretheorem[style=remark,shaded={rulecolor=SpringGreen,
rulewidth=2pt, bgcolor={rgb}{1,1,1}}]{exercise}

%Solution referring to exercise
\declaretheorem[style=remark,shaded={rulecolor=SpringGreen,
rulewidth=2pt, bgcolor={rgb}{1,1,1}}]{solution}

%%%
%
%   MARGIN NOTES SETUP
%
%%%

\usepackage{marginnote}

%%%
%
%   PGFPLOT SETUP
%
%%%

\usepackage{pgfplots}
\pgfplotsset{width=8cm, compat=1.18}

\usepackage{cleveref}

%%%
%
%   TITLE PAGE HEADER
%
%%%

\renewcommand\epigraphflush{flushright}
\renewcommand\epigraphsize{\normalsize}
\setlength\epigraphwidth{0.7\textwidth}

\definecolor{titlepagecolor}{cmyk}{1,.60,0,.40}

\DeclareFixedFont{\titlefont}{T1}{ppl}{b}{it}{0.5in}

\makeatletter                       
\def\printauthor{%                  
    {\large \@author}}              
\makeatother
\author{%
    \begin{align*}
        \text{Dario Loi, Student, Department of Computer Science}& \\
        \text{Applied Computer Science \& Artificial Intelligence}& \\
        \texttt{loi.1940849@studenti.uniroma1.it}\vspace{20pt}&
    \end{align*}
    }

% The following code is borrowed from: https://tex.stackexchange.com/a/86310/10898

\newcommand\titlepagedecoration{%
\begin{tikzpicture}[remember picture,overlay,shorten >= -10pt]

\coordinate (aux1) at ([yshift=-15pt]current page.north east);
\coordinate (aux2) at ([yshift=-410pt]current page.north east);
\coordinate (aux3) at ([xshift=-4.5cm]current page.north east);
\coordinate (aux4) at ([yshift=-150pt]current page.north east);

\begin{scope}[titlepagecolor!40,line width=12pt,rounded corners=12pt]
\draw
  (aux1) -- coordinate (a)
  ++(225:5) --
  ++(-45:5.1) coordinate (b);
\draw[shorten <= -10pt]
  (aux3) --
  (a) --
  (aux1);
\draw[opacity=0.6,titlepagecolor,shorten <= -10pt]
  (b) --
  ++(225:2.2) --
  ++(-45:2.2);
\end{scope}
\draw[titlepagecolor,line width=8pt,rounded corners=8pt,shorten <= -10pt]
  (aux4) --
  ++(225:0.8) --
  ++(-45:0.8);
\begin{scope}[titlepagecolor!70,line width=6pt,rounded corners=8pt]
\draw[shorten <= -10pt]
  (aux2) --
  ++(225:3) coordinate[pos=0.45] (c) --
  ++(-45:3.1);
\draw
  (aux2) --
  (c) --
  ++(135:2.5) --
  ++(45:2.5) --
  ++(-45:2.5) coordinate[pos=0.3] (d);   
\draw 
  (d) -- +(45:1);
\end{scope}
\end{tikzpicture}%
}

%%%
%
%   FANCY HEADER DEFINITION
%
%%%

\usepackage{fancyhdr, lastpage}
\usepackage{titlesec}
\usepackage{emptypage}
\pagestyle{fancy}

\fancyhead[LE,RO]{Dario Loi}
\fancyhead[RE,LO]{Business \& Computer Science}

\fancyfoot[CE,CO]{\leftmark}
\fancyfoot[LE,RO]{Page \thepage}

\renewcommand{\headrulewidth}{0.5pt}

\fancypagestyle{plain}{
  \renewcommand{\headrulewidth}{0pt}
  \fancyhf{}
  \fancyfoot[CE,CO]{\leftmark}
  \fancyfoot[LE,RO]{}
}

%%%
%
% HYPERREF SETUP
%
%%%

% Leave it out if you wanna look formal
%\DeclareTextFontCommand{\emph}{\bfseries}

\usepackage{hyperref}
  \hypersetup{
      colorlinks=true,
      linkcolor=blue,
      filecolor=magenta,      
      urlcolor=cyan,
      pdftitle={Business \& Computer Science},
      pdfpagemode=,
      }

%%%
%
%   TITLE PAGE MACRO
%
%%%

\newcommand\mktitlepage{
    \begin{titlepage}
        \noindent
        \titlefont Business \& Computer Science \par
        \epigraph{INSERT A QUOTE
        %
        \leavevmode %IF ENUMERATE || ITEMIZE
        %
        }%
        {\textit{CITY DATE}\\ \textsc{QUOTE AUTHOR}}
        \null\vfill
        \vspace*{1cm}
        \noindent
        \hfill
        \begin{minipage}{0.35\linewidth}
            \begin{flushright}
                \printauthor
            \end{flushright}
        \end{minipage}
        %
        \begin{minipage}{0.02\linewidth}
            \rule{1pt}{125pt}
        \end{minipage}
        \titlepagedecoration
    \end{titlepage}
}

\usepackage[a4paper, left = 3.0cm, right=3.0cm, bottom = 2.0cm, top = 4.0cm, marginparwidth=0.2cm, marginparsep=2mm, heightrounded, twoside=true]{geometry}


%%%
%
% Macroes
%
%%%

\newcommand{\sequence}[2]{
        \{ #1_1, #1_2, \dots, #1_#2 \}
}

%Common Sets
\newcommand{\R}{\mathbb{R}}
\newcommand{\N}{\mathbb{N}}
\newcommand{\Z}{\mathbb{Z}}

%insert code snippet
\NewDocumentCommand{\code}{v}{%
\texttt{#1}%
}

\newcommand{\chapterquote}[3]{ % Quote, Author, Date %
    \begin{flushleft}
            \textit{
                \`\`#1\`\`
            }
        \end{flushleft}
        \begin{flushright} 
            --- #2, \textit{#3}
        \end{flushright}
}

\begin{document}
    
    \mktitlepage
    
    
    \tableofcontents
    \newpage

    \newgeometry{a4paper, outer = 2.2cm, inner=3.4cm, bottom = 2.5cm, top = 2.5cm, marginparwidth=2.2cm, marginparsep=2mm, heightrounded, twoside=true}

    \chapter{Course Introduction}

    \section{Course Aims} 
    
    Despite what the name might make you think, the course is \emph{still} a practical one, it aims to introduce how 
    computers work in a real world scenario, such as a company.

    \paragraph{The Purpose of a Computer} 
    Through the lens of this course, the purpose of a software is to improve the company's efficency and throughput, everything 
    is evaluated through a framework of costs and profits, hence, even something as simple as buying a pen must be evaluated
    at multiple levels, from the cost of the pen itself, to the cost of the time spent by the employee to go to the store and buy it.

    \section{Course Structure}
    
    The course is usually organized in two parts, alternating on a weekly basis:

    \begin{table}[h!]
      \rowcolors{2}{gray!25}{white}
      \centering
      \begin{tabular}{c c}
          \rowcolor{gray!50}
          % Header
          $Day$ &  $Purpose$ \\
         %End Header
          Tuesday &  Lecture \\
          Thursday & Seminars\\
      \end{tabular}
      \caption{Course Structure}
      \label{tab:struct}
    \end{table}
    
    Naturally, the seminars are \emph{included} in the course's materials and are hence expected

    The exam is also split into multiple sections:

    \begin{table}[h!]
      \rowcolors{2}{gray!25}{white}
      \centering
      \begin{tabular}{c c}
          \rowcolor{gray!50}
          % Header
          Part & Contents  \\
         %End Header
          Written & 30 to 50 closed questions  \\
          Oral & Optional, to raise marks \\
      \end{tabular}
      \caption{Exam Structure}
      \label{tab:label}
    \end{table}

    The general information about the course is available at the \href{http://wwwusers.di.uniroma1.it/~cilli/}{official course page}.


    \section{Information Systems} 
    
    As studied in the first unit of \emph{Data Management \& Analysis}, an information system is a set of components that allows an organization 
    to collect, process, store and distribute information, it is \emph{not} necessarily something software-based, it can even be 
    a file cabinet!

    \begin{definition}\label{def:is}
      An Information System is something that \emph{Manages} the flow of information in an organization.
    \end{definition}

    \paragraph{Information} 
    Naturally, information is something \emph{abstract} and \emph{immaterial}, it is naturally difficult to manage, but it is possible 
    to construct a set of rules, conventions and tools that allow us to represent it in a way that is \emph{manageable}, this process 
    is the \emph{Flow}, and hence, as defined above, these systems are called \emph{Information Systems}.

    \paragraph{Computers}
    When a computer is involved, the Information System is naturally labeled a Computer Information System, or \emph{CIS} for short. 
    

    This chapter aims to give us the following capabilities

    \begin{itemize}
      \item Define what an Information System is
      \item Describe the history of Information Systems
      \item Describe the basic argument behind the article \emph{Does IT matter?} by Nicholas Carr. 
    \end{itemize}

    An information system is made up of \emph{Three} main components:

    \begin{itemize}
      \item People: In charge of decision making and organization
      \item Technology: Hardware and Software that supports the business
      \item Processes: Collecting and storing information
    \end{itemize}

    \paragraph{In General} 
    This is a pretty broad definition, that allows a wide variety of software to be seen through the lens of an Information System, 
    from network analyzers, to national healthcare systems, to online education platforms.

    \paragraph{Acronyms} 
    Depending on the application, an Information System can be called by different acronyms, such as:

    \begin{itemize}
      \item \textbf{MRP}: \emph{Manufacturing Resource Planning}
      \item \textbf{CIM}: \emph{Computer Integrated Manufacturing}
      \item \textbf{SAP}: \emph{Systems, Applications and Products}
    \end{itemize}
    
    \subsection{Anthony's Triangle} 
    
    Anthony's triangle is a diagram that categorizes the three purposes of an Information System:

    \begin{enumerate}
      \item Strategic (\emph{Executive Information System}): for senior management decisions 
      \item Tactical (\emph{Management Information System}): For middle management decisions
      \item Operational (\emph{Transaction Processing Systems}): For daily transactions of business
    \end{enumerate}

    \subsection{Information System Components} 
    Several Components work together to add value to an organization, they are 

    \begin{enumerate}
      \item Hardware: The physical components of the system
      \item Software: The programs that run on the hardware, they can be:
      \begin{itemize}
        \item Operating Systems: The programs that manage the hardware
        \item Application Software: The programs that are used by the users
      \end{itemize}
      \item Data: The information that is stored in the system
      \item People: The users of the systems, both producers and consumers of infomation
      \item Processes: The procedures that are used to collect, process and store information\footnote{
        Organizing something in \emph{processes} that is, a series of well-defined steps, brings about a series of 
        benefits in productivity.
      }
    \end{enumerate}

    This is somewhat redundant to what we stated beforehand in \cref{def:is}, but it is important to restate it to 
    emphasize the fact that the system is composed of both \emph{People} and \emph{Technology}.

    \subsection{Processes}
    
    One of the most important components of an Information System is the \emph{Process}, it is 
    the goal of the Computer Information System to optimize the processes of an organization,
    bringing about an increase in efficency.

    \paragraph{Processes Formally} 
    Since processes are an advantageous way to organize work, it is important to spend some time to 
    also give a formal definition of their nature.

    \begin{definition}
      \label{def:proc}
      A process is a series of well-defined steps that are used to achieve a specific goal, it can 
      be defined as a set of \emph{activities}
    \end{definition}

    \section{Does IT Matter?}
    Nicholas Carr, in \emph{Harvard Business Review}, argues that Information Technology is not an \emph{Investment}, 
    but rather a commodity, so something that must be managed to optimize the company's profits by reducing its 
    operational costs.
    
    \paragraph{IT as a Marketing Tool}
    It is also interesting to note how a company is percieved as better when it uses IT, and when this IT is 
    of high quality, therefore IT can be seen as a \emph{Marketing Tool} that can be used to attract customers.
    

    
\end{document}