\documentclass[openright, twoside, twocolumn]{report}
\usepackage{amsmath}
\usepackage[usenames,dvipsnames,table]{xcolor}
\usepackage{tikz}
\usepackage{quotchap}
\usepackage{epigraph}
\usepackage{etoolbox}
\usepackage{lipsum}
\usepackage[T1]{fontenc}
\usepackage{xparse}
\usepackage[binary-units]{siunitx}
\usepackage{hyperref}
\usepackage{amsfonts}
\usepackage{amsthm}
\usepackage{bookmark}
\usepackage[para]{footmisc}
\usepackage{thmtools}
\usepackage{booktabs}


%────────────────────────────────────────────────────────────────────────────────────────────────────────────────────────────────────────────────────
%
% Super Figure preamble.
%
%────────────────────────────────────────────────────────────────────────────────────────────────────────────────────────────────────────────────────

% generated by the Super Figure vscode extension. May we stand on the shoulder's of giants (RIP Gilles.)
\usepackage{import}
\newcommand{\incsvg}[2]{%
    \def\svgwidth{\columnwidth}
    \graphicspath{{#1}}
    \input{#2.pdf_tex}
}


%────────────────────────────────────────────────────────────────────────────────────────────────────────────────────────────────────────────────────
%
% Listings (Python)
%
%────────────────────────────────────────────────────────────────────────────────────────────────────────────────────────────────────────────────────

\usepackage{listings}


\definecolor{codegreen}{rgb}{0,0.6,0}
\definecolor{codegray}{rgb}{0.5,0.5,0.5}
\definecolor{codepurple}{rgb}{0.58,0,0.82}
\definecolor{backcolour}{rgb}{0.95,0.95,0.92}

\lstdefinestyle{mystyle}{
    backgroundcolor=\color{backcolour},
    commentstyle=\color{codegreen},
    keywordstyle=\color{magenta},
    numberstyle=\tiny\color{codegray},
    stringstyle=\color{codepurple},
    basicstyle=\footnotesize,
    breakatwhitespace=false,
    breaklines=true,
    captionpos=b,
    keepspaces=true,
    numbers=left,
    numbersep=5pt,
    showspaces=false,
    showstringspaces=false,
    showtabs=false,
    tabsize=2
}

\lstset{style=mystyle}
\lstset{language=Python}

%────────────────────────────────────────────────────────────────────────────────────────────────────────────────────────────────────────────────────
%
% Theorems
%
%────────────────────────────────────────────────────────────────────────────────────────────────────────────────────────────────────────────────────


% Main theorem style
\declaretheorem[shaded={bgcolor={White!90!Periwinkle}}, numberwithin=section]{theorem,definition}
\declaretheorem[shaded={bgcolor={White!90!Periwinkle}}, numberwithin=section]{lemma,corollary,proposition}


\declaretheorem[shaded={bgcolor={White!90!Periwinkle}}, numberwithin=section]{remark,observation}

\declaretheorem[style=remark,shaded={bgcolor={White!90!Periwinkle}}, numberwithin=section]{example}

\declaretheorem[style=remark,shaded={bgcolor={White!90!Periwinkle}}, numberwithin=section]{exercise}

%Solution referring to exercise
\declaretheorem[style=remark,shaded={bgcolor={White!90!Periwinkle}}, numberwithin=section]{solution}

%%%
%
%   MARGIN NOTES SETUP
%
%%%

\usepackage{marginnote}

%%%
%
%   PGFPLOT SETUP
%
%%%

\usepackage{pgfplots}
\pgfplotsset{width=8cm, compat=1.18}

\usepackage{cleveref}

%%%
%
%   TITLE PAGE HEADER
%
%%%

\renewcommand\epigraphflush{flushright}
\renewcommand\epigraphsize{\normalsize}
\setlength\epigraphwidth{0.7\textwidth}

\definecolor{titlepagecolor}{cmyk}{1,.60,0,.40}

\DeclareFixedFont{\titlefont}{T1}{ppl}{b}{it}{0.5in}

\makeatletter
\def\printauthor{%
    {\large \@author}}
\makeatother
\author{%
    \begin{align*}
        \text{Dario Loi, Student, Department of Computer Science}& \\
        \text{Applied Computer Science \& Artificial Intelligence}& \\
        \texttt{loi.1940849@studenti.uniroma1.it}\vspace{20pt}&
    \end{align*}
    }

% The following code is borrowed from: https://tex.stackexchange.com/a/86310/10898

\newcommand\titlepagedecoration{%
\begin{tikzpicture}[remember picture,overlay,shorten >= -10pt]

\coordinate (aux1) at ([yshift=-15pt]current page.north east);
\coordinate (aux2) at ([yshift=-410pt]current page.north east);
\coordinate (aux3) at ([xshift=-4.5cm]current page.north east);
\coordinate (aux4) at ([yshift=-150pt]current page.north east);

\begin{scope}[titlepagecolor!40,line width=12pt,rounded corners=12pt]
\draw
  (aux1) -- coordinate (a)
  ++(225:5) --
  ++(-45:5.1) coordinate (b);
\draw[shorten <= -10pt]
  (aux3) --
  (a) --
  (aux1);
\draw[opacity=0.6,titlepagecolor,shorten <= -10pt]
  (b) --
  ++(225:2.2) --
  ++(-45:2.2);
\end{scope}
\draw[titlepagecolor,line width=8pt,rounded corners=8pt,shorten <= -10pt]
  (aux4) --
  ++(225:0.8) --
  ++(-45:0.8);
\begin{scope}[titlepagecolor!70,line width=6pt,rounded corners=8pt]
\draw[shorten <= -10pt]
  (aux2) --
  ++(225:3) coordinate[pos=0.45] (c) --
  ++(-45:3.1);
\draw
  (aux2) --
  (c) --
  ++(135:2.5) --
  ++(45:2.5) --
  ++(-45:2.5) coordinate[pos=0.3] (d);
\draw
  (d) -- +(45:1);
\end{scope}
\end{tikzpicture}%
}

%%%
%
%   FANCY HEADER DEFINITION
%
%%%

\usepackage{fancyhdr, lastpage}
\usepackage{titlesec}
\usepackage{emptypage}
\pagestyle{fancy}

\fancyhead[LE,RO]{Dario Loi}
\fancyhead[RE,LO]{Human Computer Interaction}

\fancyfoot[CE,CO]{\leftmark}
\fancyfoot[LE,RO]{Page \thepage}

\renewcommand{\headrulewidth}{0.5pt}

\fancypagestyle{plain}{
  \renewcommand{\headrulewidth}{0pt}
  \fancyhf{}
  \fancyfoot[CE,CO]{\leftmark}
  \fancyfoot[LE,RO]{}
}

%%%
%
% HYPERREF SETUP
%
%%%

\usepackage{hyperref}
  \hypersetup{
      colorlinks=true,
      linkcolor=blue,
      filecolor=magenta,
      urlcolor=cyan,
      pdftitle={Human Computer Interaction},
      pdfpagemode=,
      }

%%%
%
%   TITLE PAGE MACRO
%
%%%

\newcommand\mktitlepage{
    \begin{titlepage}
        \noindent
        \titlefont Human Computer Interaction \par
        \epigraph{INSERT A QUOTE
        %
        \leavevmode %IF ENUMERATE || ITEMIZE
        %
        }%
        {\textit{CITY DATE}\\ \textsc{QUOTE AUTHOR}}
        \null\vfill
        \vspace*{1cm}
        \noindent
        \hfill
        \begin{minipage}{0.35\linewidth}
            \begin{flushright}
                \printauthor
            \end{flushright}
        \end{minipage}
        %
        \begin{minipage}{0.02\linewidth}
            \rule{1pt}{125pt}
        \end{minipage}
        \titlepagedecoration
    \end{titlepage}
}

\usepackage[a4paper, left = 3.0cm, right=4.8cm, bottom = 2.5cm, top = 4.5cm, marginparwidth=2.2cm, marginparsep=2mm, heightrounded, twoside=true]{geometry}


%%%
%
% Macroes
%
%%%

\newcommand{\sequence}[2]{
        \{ #1_1, #1_2, \dots, #1_#2 \}
}

%Common Sets
\newcommand{\R}{\mathbb{R}}
\newcommand{\N}{\mathbb{N}}
\newcommand{\Z}{\mathbb{Z}}

%insert code snippet
\NewDocumentCommand{\code}{v}{%
\texttt{#1}%
}

\newcommand{\chapterquote}[3]{ % Quote, Author, Date %
    \begin{flushleft}
            \textit{
                \`\`#1\`\`
            }
        \end{flushleft}
        \begin{flushright}
            --- #2, \textit{#3}
        \end{flushright}
}

\begin{document}

    \mktitlepage


    \tableofcontents
    \newpage

    \newgeometry{a4paper, outer = 2.2cm, inner=3.4cm, bottom = 2.5cm, top = 2.5cm, marginparwidth=2.2cm, marginparsep=2mm, heightrounded, twoside=true}

    \chapter{Need Finding}

    In order to develop a good application, we need to understand the needs of our users, during this first lectures we will focus on a variety of methods and procedures with which to achieve this goal.

    \section{Observing}

    A good way to understand the needs of our users is to observe them, this allows us to identify existing problems, understand their goals and their needs, observation must be conducted:

    \begin{itemize}
      \item Without inductive bias (we don't pretend that we already have the solution)
      \item With a mind open to unexpected discoveries
      \item Knowing that it is observation that allows us to define the problem itself
    \end{itemize}

    We want to observe not only the user and their actions, but also the environment,
    the tools at hand, the context, the social interactions, etc.

    \begin{remark}
      User perception is often wrong! pay attention to what they \emph{say} and what they \emph{do}, they are often not the same.
    \end{remark}

    \begin{remark}
      Do \emph{not} confuse needs and solutions, as said before, do not try to find a solution to a problem, but try to understand the problem itself, otherwise you risk limiting your creativity.
    \end{remark}

    \section{Techniques}
    In order to gather the information we need, we can go further than just observing, some other techniques are:

    \begin{itemize}
      \item Interviews -- Ask people questions
      \item Questionnaires -- Ask people to fill out a form
      \item Room studies -- Observe people in their natural environment
      \item Pager studies -- Ask people to carry a pager and record their activities
      \item Diary -- Ask people to keep a diary of their activities
    \end{itemize}

    We focus on \emph{interviews} since they are \textcolor{ForestGreen}{informal}, \textcolor{ForestGreen}{cheap}, and allow us to obtain a lot of \textcolor{ForestGreen}{unexpected information}, unfortunately they are also
    \textcolor{Red}{subjective} and \textcolor{Red}{time consuming}.

    \paragraph{User Groups}
    When interviewing, we must realize that different people have different approaches to a product, some might be expert users, some might be casual users, some might not even be users at all, hence we must be careful to understand the different groups of users when interviewing.

    \subsection{Guidelines}
    In general, there are some good practices to follow when interviewing:

    \begin{itemize}
      \item Prepare the questions first
      \item Record the interview (or take notes)\footnote{
        Remember to ask for permission before recording the interview
      }
      \item Be polite and friendly
      \item Encourage the interviewee to elaborate on their answers
      \item Remain neutral towards the interviewee's positions and opinions
    \end{itemize}

    \paragraph{Questions to Beware}
    There are some questions that should be avoided, since they are either too general or too specific,
    here is a list of some of them:

    \begin{itemize}
      \item Questions with an obvious answer -- focus on the \emph{why}
      \item Questions that contain the answer -- try to keep the questions open-ended to allow the interviewee to elaborate
      \item Questions that are too general -- try to be as practical as possible
      \item Questions that ask the user to replace the designer -- `Would you like this function?' rather than `How do you think this function should work?'
      \item Questions based on weird hypotheticals -- \emph{Keep it real.}
      \item Questions that ask how often something happens -- Indicate when exactly it happens.
    \end{itemize}

    \subsection{Questionnaires}
    As previously explained, there are different types of users, if we want to avoid tailoring interviews toward each group, we can use questionnaires, which are a good way to gather information from a large number of people.

    Questionnaires are \textcolor{ForestGreen}{fast}, can be analyzed more \textcolor{ForestGreen}{easily}, and can be distributed to a \textcolor{ForestGreen}{large} number of people, however they are also \textcolor{Red}{less flexible}.

    \paragraph{Questionnaire Design}
    When designing a questionnaire, we must keep in mind that we want to obtain \emph{precise} and \emph{reliable} information, we must take great care into avoiding \emph{bias} and \emph{ambiguity}, especially from the questions themselves.

    The professor suggests \href{https://www.google.it/intl/it/forms/about/}{Google Forms} as a good tool to create questionnaires.


    \chapter{Storyboarding}

    \section{Task Analysis}

    Up until now, there haven't been any notes since the lectures were largely focused on requirement analysis, we
    now want to understand a few things:

    \begin{itemize}
      \item Which needs we want to satisfy?
      \item Why should people use our application?
      \item What should our application allow to do?
    \end{itemize}

    \begin{remark}
        You are \emph{Not} the user of your application, hence try to obtain external opinions
        on your user experience.
    \end{remark}

    \paragraph{Tasks}

    We want to identify the set of \emph{tasks} supported in our application

    \begin{definition}
        A task is a sequence of actions that a user performs to satisfy its needs, the task
        is part of a general \emph{activity}
    \end{definition}

    Hence, in this phase, we are mainly interested in the role of the \emph{user interface},
    not only on a functionality level, but also in relation to its users, for example:

    \begin{itemize}
      \item The actors involved
      \item The environment
      \item The required tasks
    \end{itemize}

    We are \emph{not} designing the screens yet, we are just trying to understand the role
    of the user interface in the application.

    \subsection{Drawing The Tasks}

    It is optimal to provide a pictorial representation of the flow of the tasks, this helps
    understanding the way the user approaches the application, when doing so, we are under
    these constraints:

    \begin{itemize}
      \item Efficiency -- We want a quick mock-up
      \item Comprehensibility -- We want it to be understandable
      \item Communicating the tasks to the designers
    \end{itemize}

    The produced drawing is called a \textbf{storyboard} and it is a sequence of pictures
    that represent the flow of the tasks, to its core, it is a \textbf{comic}.

    The perfect storyboard is:

    \begin{itemize}
      \item Hand drawn
      \item Clear and simple
      \item With as few text as possible
      \item With as few panels as possible
    \end{itemize}

    \paragraph{What to draw?}
    The storyboard should provide \emph{snapshots} of the interface at particular points
    during the user interaction, it should be as simple as possible, but personal notes
    to clarify the drawings are allowed and encouraged when they don't clutter the drawing.

    It is important to not draw the screen, since at this stage in development we don't know
    what the screen should look like, we just want to understand the flow of the tasks, this
    allows us to be unconstrained by implementation details.

    \paragraph{How Many To Draw?}

    One should draw one storyboard for each task, initially, we want to only consider
    the \emph{main} tasks, doing \emph{everything} is a bad idea, since it reduces
    the agility of the design process, we want to focus on the \emph{core} of our application.

\end{document}